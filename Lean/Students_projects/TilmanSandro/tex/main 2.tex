\documentclass{beamer}
\usetheme{Copenhagen}


\usepackage{appendixnumberbeamer}

\usepackage{booktabs}


\usepackage[utf8]{inputenc}
\usepackage[english]{babel}
\usepackage[T1]{fontenc}
\usepackage{multirow}
\usepackage{amsmath}
\usepackage{graphicx}
\graphicspath{ {./pictures/} }
\usepackage{tikz}
\usetikzlibrary{shapes.geometric, arrows}
\usepackage{subcaption}

\usepackage{pdfpages}
\setbeamertemplate{footline}[frame number]
\usepackage{listings}
\usepackage{amssymb}
\usepackage{color}
\definecolor{keywordcolor}{rgb}{0.7, 0.1, 0.1}   % red
\definecolor{tacticcolor}{rgb}{0.0, 0.1, 0.6}    % blue
\definecolor{commentcolor}{rgb}{0.4, 0.4, 0.4}   % grey
\definecolor{symbolcolor}{rgb}{0.0, 0.1, 0.6}    % blue
\definecolor{sortcolor}{rgb}{0.1, 0.5, 0.1}      % green
\definecolor{attributecolor}{rgb}{0.7, 0.1, 0.1} % red

\def\lstlanguagefiles{lstlean.tex}
% set default language
\lstset{language=lean}

\title{The Fundamental Theorem of Algebra via Linear Algebra}
\subtitle{}
\date{July 2023}
\author{Tilman Jacobs, Sandro Reinhard}
\institute{Seminar on Computer Assisted Mathemathics}

\begin{document}

\maketitle

%%%%%%%%%%%%%%%%%%%%%%%%%%%%%%%%%%%%%%%%%%%%%%%%%%%%%%%%%%%%%%%%%%%%%%%%
% Introduction
%%%%%%%%%%%%%%%%%%%%%%%%%%%%%%%%%%%%%%%%%%%%%%%%%%%%%%%%%%%%%%%%%%%%%%%%
\begin{frame}
    \begin{Theorem}[Fundamental Theorem of Algebra]
              Any nonconstant polynomial with coefficients in $\mathbb{C}$ has a complex root.
    \end{Theorem}
  \begin{proof}[Usual Proof via Complex Analysis]
      Apply Liouville's theorem to show that the reciprocal of a polynomial without roots is bounded and therefore already constant.
  \end{proof}
  However, since the FTA is a statement mostly associated with Linear Algebra we are interested in a proof via LA...
\end{frame}


\begin{frame}[fragile]
    \begin{theorem}[Restatement of FTA]
        For each $n \geq 1$, every $n \times n$ square matrix over $\mathbb{C}$ has an eigenvector in $\mathbb{C}^n$. Equivalently, for each $n \geq 1$, every linear operator on an $n$-dimensional complex vector space $V$ has an eigenvector in $V$ .
    \end{theorem}
    \hfill
        \begin{lstlisting}
variable {m : ℕ} [Fintype (Fin m)] [Field ℂ]

def IsEigenvector (A : Matrix (Fin m) (Fin m) ℂ) (v : Fin m → ℂ) := (v ≠ 0) ∧ (∃ μ : ℂ, (mulVec A v) = μ • v)

theorem exists_eigenvector (A : Matrix (Fin m) (Fin m) ℂ) : (m ≥ 1) → (∃ v : Fin m → ℂ, IsEigenvector A v) :=
by sorry


    \end{lstlisting}
\end{frame}



%%%%%%%%%%%%%%%%%%%%%%%%%%%%%%%%%%%%%%%%%%%%%%%%%%%%%%%%%%%%%%%%%%%%%%%%
% Math
%%%%%%%%%%%%%%%%%%%%%%%%%%%%%%%%%%%%%%%%%%%%%%%%%%%%%%%%%%%%%%%%%%%%%%%%
\begin{frame}[fragile]
    \begin{lemma}
       Fix an integer $n\geq 1$ and a field $\mathbb{F}$. Suppose that, for very $\mathbb{F}$-vector space $V$ whose dimension is not divisible by $m$, every linear operator on V has an eigenvector in $V$. Then for every $\mathbb{F}$-vector space $V$ whose dimension is not divisible by $m$, each pair of commuting linear operators has a common eigenvector in $V$.
    \end{lemma}
    
    \hfill
\begin{lstlisting}[language=Lean, basicstyle=\scriptsize]
    
 universe u v w
 
 variable {ℂ : Type v} {V : Type w} [Field ℂ] [AddCommGroup V] [Module ℂ V]
 
 lemma comm_lin_opHasEigenvector [FiniteDimensional ℂ V] [Nontrivial V] (f : End ℂ V) (g : End ℂ V) (h : End ℂ V) : (m ≥ 1) ∧ ¬(m | (finrank ℂ  V))∧ (∃ v : V , f.HasEigenvector μ v)→ (∃ v : V , g.HasEigenvector μ v ∧ h.HasEigenvector ν v)  := sorry


\end{lstlisting}
\end{frame}

\begin{frame}[fragile]
  \begin{corollary}
      For every real vector space $V$ whose dimension is odd each pair of of commuting linear operators on $V$ has a common eigenvector in $V$.
  \end{corollary}  

\begin{lstlisting}[language=Lean, basicstyle=\scriptsize]
theorem exists_eigenvector_odd (A : Matrix (Fin m) (Fin m) ℂ): (Odd(card)) → (∃ v :  Fin m → ℂ, IsEigenvector  A v) := by sorry
\end{lstlisting}

\end{frame}
\begin{frame}
\begin{block}{Main idea}
    Strong Induction on the highest power of 2 dividing dimension $n$ of $V$
\end{block}
    $$n=2^kn':k\geq0,n' \text{odd}$$
\end{frame}
%\begin{frame}{Base case ($k=0$)}
    %Let $n$ be odd, $A\in M_n(\mathbb{C})$
%\end{frame}
\end{document}